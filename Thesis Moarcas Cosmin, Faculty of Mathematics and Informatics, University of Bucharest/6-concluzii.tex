\chapter{Concluzii}

În această lucrare, am demonstrat capabilitatea rețelelor neurale spike de a oferi rezultate comparabile cu modelele convenționale de învățare automată, având avantajul unui consum redus de energie în timpul antrenării. Am evaluat eficiența rețelelor neurale în clasificarea accentelor folosind trei arhitecturi distincte: o rețea neurală convoluțională antrenată cu amplitudinile semnalelor, o rețea neurală convoluțională antrenată cu coeficienții cepstrali și o rețea neurală spike antrenată cu coeficienții cepstrali. Fiecare rețea a fost evaluată atât din punct de vedere al performanței, cât și al consumului energetic.

În ceea ce privește performanța, cele trei arhitecturi au demonstrat rezultate remarcabile, cu diferențe relativ mici între ele. Cele mai bune rezultate au fost obținute de rețeaua convoluțională antrenată cu amplitudinile semnalelor, oferind cea mai mare acuratețe, de 98.4\%. Următoarea a fost rețeaua convoluțională antrenată cu coeficienții cepstrali, cu o acuratețe de 98.3\%. Deși rețeaua neurală spike a obținut cea mai mică acuratețe, de 96.5\%, aceasta a fost foarte apropiată de primele două.

Din punct de vedere al eficienței energetice, evaluările au arătat următoarele rezultate: 70,875 Jouli pentru antrenarea rețelei convoluționale cu amplitudinile semnalului audio, 910,043.5 Jouli pentru antrenarea rețelei convoluționale cu coeficienți cepstrali și doar 44.8 Jouli pentru antrenarea rețelei neurale spike. Aceste cifre evidențiază avantajul semnificativ al rețelelor neurale spike în ceea ce privește consumul de energie. Mai precis, rețeaua neurală spike este de aproximativ 1581 de ori mai eficientă energetic decât rețeaua convoluțională antrenată cu amplitudinile semnalului audio și de aproximativ 20313 de ori mai eficientă decât rețeaua convoluțională antrenată cu coeficienți cepstrali.

În concluzie, am demonstrat că rețelele neurale spike reprezintă o alternativă viabilă și sustenabilă la modelele convenționale de învățare automată, oferind un echilibru excelent între performanță și eficiență energetică. Aceste rețele deschid noi oportunități pentru cercetare și aplicabilitate practică într-o gamă largă de domenii tehnologice, promițând soluții inovatoare și mai eficiente energetic.





