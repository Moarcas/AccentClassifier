\chapter{Introducere}

Link GitHub: \href{https://github.com/Moarcas/licenta}{https://github.com/Moarcas/licenta}


\section{Contextul și importanța clasificării accentului}

Accentul reprezintă modul specific în care o persoană pronunță cuvintele unei limbi. Fiecare vorbitor al unei limbi are un 
accent distinct, influențat de factori geografici, culturali și sociali. La nivel mondial, există aproximativ 160 de accente 
diferite, reflectând diversitatea lingvistică și culturală globală.

Studierea accentului este deosebit de importantă datorită legăturilor sale strânse cu multiple probleme sociale, precum 
acceptarea vorbitorului într-o comunitate și percepția acestuia în cadrul unei anumite clase sociale. De exemplu, accentul 
poate influența modul în care o persoană este percepută din punct de vedere profesional și social, afectându-i inclusiv 
oportunitățile de angajare și integrarea socială.

Un caz particular relevant este cel al tinerilor din India, unde engleza a devenit o limbă frecvent utilizată datorită 
prezenței sale în sistemul educațional. Majoritatea tinerilor vorbesc cel puțin două limbi: una dintre cele 22 de limbi 
oficiale ale Indiei și limba engleză. Această bilingvism larg răspândit, combinat cu diversitatea accentelor, face ca engleza 
vorbită în India să fie o resursă valoroasă pentru studierea accentului.

Analizarea accentului în acest context poate contribui semnificativ la dezvoltarea unor sisteme de recunoaștere vocală mai 
precise și adaptate, care să funcționeze eficient într-un mediu multilingv și multicultural. De asemenea, poate facilita 
îmbunătățirea asistenților vocali, a traducerilor automate și a altor aplicații ce depind de procesarea limbajului natural.

Prin urmare, clasificarea accentului nu este doar o problemă tehnică, ci și una cu implicații sociale profunde, având 
potențialul de a îmbunătăți interacțiunile dintre oameni și tehnologie într-un mod mai incluziv.


\section{Obiectivele lucrării}

Obiectivul principal al acestei lucrări este de a demonstra eficiența energetică a rețelelor neurale spike în clasificarea a 
nouă accente distincte. În detaliu, acest obiectiv se va atinge prin:
\begin{enumerate}
    \item \textbf{Preprocesarea datelor audio:} vom prelucra datele audio prin diverse tehnici pentru a pregăti setul de date 
    utilizat în antrenarea rețelelor neurale.
    \item \textbf{Dezvoltarea unei rețele neurale spike:} vom antrena rețele neurale spike utilizând biblioteca snnTorch din 
    Python.
    \item \textbf{Dezvoltarea unei rețele neurale artificiale:} vom antrena rețele neurale artificiale tradiționale utilizând 
    biblioteca Torch din Python. Aceste rețele vor servi ca punct de referință pentru comparația performanțelor și consumului 
    energetic cu rețelele neurale spike.
    \item \textbf{Evaluarea performanței pentru fiecare rețea:} vom analiza performantele fiecarei retea utilizand metrici 
    precum: acuratetea, matricea de confuzie, scorul F1. Aceasta va permite o evaluare detaliată a capacității fiecărei rețele 
    de a clasifica corect accentele.
    \item \textbf{Analiza performanței energetice:} Vom compara consumul de energie al rețelelor neurale spike cu cel al 
    rețelelor neurale tradiționale. Această comparație va include măsurători detaliate ale consumului energetic în timpul 
    antrenării și inferenței, evidențiind avantajele potențiale ale utilizării rețelelor neurale spike din perspectiva 
    eficienței energetice.
\end{enumerate}

Prin atingerea acestor obiective specifice, lucrarea își propune să demonstreze că rețelele neurale spike nu doar că pot fi 
utilizate eficient pentru clasificarea accentelor, dar și că acestea oferă avantaje semnificative în ceea ce privește 
eficiența energetică, aspect esențial în contextul dezvoltării unor sisteme de inteligență artificială sustenabile și 
eficiente.


\section{Structura lucrării}

Structura lucrării este următoarea: capitolul "Preliminarii" furnizează informații esențiale despre rețelele neurale 
artificiale, spike și datele audio. Capitolul "Fundamentele Teoretice ale Rețelelor Neurale Spike" se concentrează pe teoria 
din spatele acestor rețele. Capitolul "Setul de Date" introduce și detaliază datele utilizate pentru antrenare și modurile în 
care acestea au fost procesate. În capitolul "Modele Dezvoltate", sunt prezentate și evaluate atât modelele de rețele neurale 
artificiale, cât și cele spike dezvoltate, analizându-se performanța și consumul de energie. Capitolul "Concluzii" 
sintetizează rezultatele și concluziile obținute în cadrul lucrării.