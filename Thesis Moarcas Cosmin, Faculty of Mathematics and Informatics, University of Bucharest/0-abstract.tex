\begin{abstractpage}

\begin{abstract}{romanian}

În ultimii ani, domeniul inteligenței artificiale a cunoscut o creștere semnificativă. Odată cu integrarea sistemelor de 
inteligență artificială în diverse domenii, consumul de energie necesar antrenării acestora a crescut considerabil.

Pentru a aborda problema consumului ridicat de energie, companii precum Intel și CyberSwarm din România au dezvoltat cipuri 
neuromorfice, menite să eficientizeze sistemele convenționale printr-o arhitectură inovatoare, inspirată din principiile de 
funcționare ale creierului. Aceste cipuri facilitează antrenarea unor rețele neuronale diferite de cele artificiale, numite 
rețele neuronale spike.

Această lucrare își propune dezvoltarea și evaluarea rețelelor neuronale spike în clasificarea accentelor în limba engleză. 
Comparând rețelele neuronale adânci cu cele spike, am demonstrat că ambele ating performanțe similare, însă cele din urmă 
oferă avantajul unui consum redus de energie în timpul antrenării. Modelele de clasificare a accentelor pot îmbunătăți 
substanțial performanța sistemelor de recunoaștere vocală precum Alexa și Siri. Prin detectarea accentului, se pot dezvolta 
sisteme specializate pentru fiecare tip de accent, facilitând utilizarea acestor tehnologii de către persoanele cu un accent 
pronunțat.

\end{abstract}

\begin{abstract}{english}

In recent years, the field of artificial intelligence has experienced significant growth. With the integration of AI systems 
across various domains, the energy consumption required for training these systems has also increased considerably.

To address the issue of high energy consumption, companies such as Intel and CyberSwarm from Romania have developed 
neuromorphic chips designed to enhance the efficiency of conventional systems through an innovative architecture inspired by 
the functioning principles of the brain. These chips facilitate the training of neural networks that are different from 
traditional artificial ones, called spiking neural networks.

This paper aims to develop and evaluate spiking neural networks for accent classification in the English language. By 
comparing deep neural networks with spiking ones, we demonstrated that both achieve similar performance levels, but the latter 
offer the advantage of reduced energy consumption during training. Accent classification models can substantially improve the 
performance of voice recognition systems like Alexa and Siri. By detecting accents, specialized systems can be developed for 
each accent type, making these technologies more accessible to individuals with pronounced accents.

\end{abstract}


\end{abstractpage}